
\documentclass[10pt,fullpage]{article}

\usepackage{hyperref}
\usepackage{geometry}
\usepackage{listliketab}
\usepackage{array}
\usepackage{longtable}

\usepackage{verbatim}
\usepackage{natbib}
\usepackage{bibentry}
\nobibliography*
%\bibliographystyle{apa}

% Palatino font
%\usepackage[T1]{fontenc}
%\usepackage[sc,osf]{mathpazo}

\def\name{Mashfiqui Rabbi}

\reversemarginpar

\geometry{
  body={6.5in, 8.5in},
  left=1.0in,
  top=1.0in
}

% Customize page headers
\pagestyle{myheadings}
\markright{\name}
\thispagestyle{empty}

% Custom section fonts
\usepackage{sectsty}
\sectionfont{\rmfamily\mdseries\Large}
\subsectionfont{\rmfamily\mdseries\itshape\large}

% Other possible font commands include:
% \ttfamily for teletype,
% \sffamily for sans serif,
% \bfseries for bold,
% \scshape for small caps,
% \normalsize, \large, \Large, \LARGE sizes.

% Don't indent paragraphs.
\setlength\parindent{1em}

% Make lists without bullets
\renewenvironment{itemize}{
  \begin{list}{}{
    \setlength{\leftmargin}{1em}
  }
}{
  \end{list}
}

\begin{document}
\bibliographystyle{abbrv}

% Print name centered and bold:
\centerline{\Large \bf \name}

%\vspace{0.25in}

\begin{minipage}{0.50\linewidth}
  \href{http://www.cornell.edu/}{Harvard University} \\
  \href{http://infosci.cornell.edu/}{Department of Statistics} \\
  Cambridge, MA 02138
\end{minipage}
\begin{minipage}{0.50\linewidth}
  \begin{tabular}{ll}
    Phone: & (603) 667-1797 \\
    Email: & \href{mailto:mrabbi@umich.edu}{mrabbi@fas.harvard.edu} \\
    Homepage: & \href{https://mashfiqui-rabbi.github.io/}{https://mashfiqui-rabbi.github.io}\\
    Google Scholar: & \href{http://bit.ly/MashGScholar}{bit.ly/MashGScholar}\\
    LinkedIn: & \href{http://bit.ly/MashLinkedIn}{bit.ly/MashLinkedIn}
  \end{tabular}
\end{minipage}

%\vspace{1em}

%General: Mobile Health, Ubiquitous and Wearable Computing, Applied Physics, Speech Interests and Audio Signal Processing, Machine Learning, Embedded and Mobile Systems, Affective Computing Specific: Understanding Human Physical and Psychological Behavior using Mobile Sensors, Sensing Human Physiology via Body Sound and Vibration, Leveraging Electromagnetic and Sound Wave for Sensing Human Body and Environment, Geo-spatial Modeling of Contagious Disease with Contactless Sensors.

\section*{\textbf{Research Interest}}
\vspace{-0.5em}
\begin{longtable}{p{0.7in}|p{5.5in}}
		General & Mobile health, Mobile computing, Machine learning, Data Science, Behavior change, Human-computer Interaction \vspace{0.15cm}\\
	Specific & Just-in-time adaptive interventions (JITAIs), Intervention design, Activity recognition, User-centered design, Causal inference on time-varying treatments, User-engagement \\
\end{longtable}

%Coming from a computer science background, 
%I have a keen interest in building systems that have social impact and at same time require computational challenges to overcome. Thus after starting my research on applying sensors to understand and influene well-being, I didn't need any further motivation to continue working in the field. %Currently in my work, I am building scalable mobile health systems that can continuously capture and reason with sensor data in. 
%Now, I am building scalable mobile health systems that continuously capture and reason on how individual users are exercising, eating, sleeping and talking.
%Based on such information, I am trying to find markers that are indicative of mental and physical well-being. Recently I started working on promoting healthier life-style by finding fine-grained behavioral markers specific to a user's well-being, and suggesting to avoid, modify or maintain such behaviors. In generating these suggestions, I am using decision theory techniques for the first time in mobile health literature with optimization criteria set as a mixture of well-being goals and how humans act in behavior change settings.
%\noident
%I am interested to build mobile health systems that use phone data (e.g., sensors or self-input) to understand human behavior and personalize intervention at the right time. In my PhD,  I built an app called MyBehavior; MyBehavior is the first app that used machine learning to understand physical activity and dietary behavior of its user, and personalized recommendations that ask for small changes.   
%In my postdoc, I am working on an app called SARA; the goal of SARA is to provide incentive at the right time to increase engagement of self-reported data on phone.  Methodologically my work focuses on creative use of machine learning and human computer interaction; machine learning is necessary to reason about data, and human computer interaction is necessary to ensure all the machine learning is actually used by real people in need.
%\vspace{1em}
\section*{\textbf{Education}}
\vspace{-1em}
\begin{longtable}{p{0.7in}|p{5.5in}}
		2017-now & {\bf Post-doctoral fellow} at Statistics Department, {\it Harvard University} \\
			& \underline{{\it Advisor}}: Susan Murphy\vspace{0.15cm}\\
	2016-2017 & {\bf Post-doctoral fellow} at Statistics Department, {\it University of Michigan, Ann Arbor} \\
			& \underline{{\it Advisors}}: Susan Murphy, Ambuj Tewari, Predrag Klasnja\vspace{0.15cm}\\	2011-2016 & {\bf Ph.D.} in Information Science, {\it Cornell University} \\
			& \underline{{\it Committee}}: Tanzeem Choudhury (chair), Deborah Estrin, Dan Cosley\\
			& Awarded Masters degree on October 2014 \\
			& Graduated April 2016\vspace{0.15cm}\\
	2009-2011& {\bf Ph.D. Student} in Computer Science, {\it Dartmouth College} \\
			& \underline{{\it Advisor}}: Tanzeem Choudhury \\
			&Transferred to Cornell University with advisor and continued PhD\vspace{0.15cm}\\
	2003-2007 & {\bf B.Sc.} in Computer Science and Engineering, {\it Bangladesh Uni. of Eng. and Tech.}\\
			& \underline{{\it CGPA}}: 3.74 out of 4.00, \underline{{\it Major}}: 3.81\\
			& \underline{{\it Thesis Advisor}}: Saidur Rahman\\
			%& \underline{{\it Thesis Topic}}: Layered drawing of planar graphs\\
\end{longtable}
%\vspace{1em}


\section*{\textbf{Work Experience}}
\vspace{-1em}
\begin{longtable}{p{0.7in}|p{5.5in}}
	2011-2016 & {\bf Research Assistant} in Department of Information Science at {\it Cornell University} \vspace{0.15cm}\\
		%& \underline{{\it Notable projects}}: $MyBehavior$, an automated and personalized health feedback system; $StreeSense$, detection of stressful interactions from face-to-face conversations; $SAINT$, a mobile sensing and inference toolkit that can handle multiple applications and can easily scale to activity recognition problems with complex inter-dependencies. $MoodRhythm$, sensing and inferring manic and depressive episodes for patients with bi-polar depression.\\
	2014 & {\bf Summer Intern} at {\it Intel Labs} \\
		& \underline{{\it Advisors}}: Lama Nachman, Hong Lu \vspace{0.15cm}\\	
		%& \underline{{\it Summary}}: I built a mobile phone based automated personalized health feedback system where a notification is posted to a user on their watch. The notification is posted when an appropriate context of following a health suggestion arise (e.g., walking suggestions while somebody comes to office). The goal is that if a user follows the suggestion in a given context again and again then the users will build up good habits in the environment they live in.\\
	2013 & {\bf Teaching Assistant} in Department of Information Science at {\it Cornell University} \vspace{0.15cm}\\
		%& \underline{{\it Responsibility}}: I helped setup and support a course on ubiquitous computing offered for the first time at Cornell University. Significant portion of the class involved hands on exercise in physical activity, localization, emotion and brain activity recognition. \\
	2012 & {\bf Summer Intern} at {\it AT\&T Labs Research} \\
		& \underline{{\it Advisors}}: Emiliano Miluzzo, Suhrid Balakrishnan \vspace{0.15cm}\\	
		%& \underline{{\it Summary}}: I worked on a mobile phone based personalized health feedback system that applies sequential decision theory concept in health feedback for the first time. This project was showcased at the yearly ``AT\&T Innovation Showcase `Connecting Your World'" in 2013. Only six projects from AT\&T research labs were shown at the event.\\
	2009-2011 & {\bf Research Assistant} in Department of Computer Science, {\it Dartmouth College} \\
			& \underline{{\it Advisors}}: Tanzeem Choudhury, Andrew Campbell \vspace{0.15cm}\\
		%& \underline{{\it Notable projects}}: $BeWell$, a persuasive system for multi-dimensional well-being (socialization, physical activity, sleep); Passive and in-situ assessment of mental (depression, sociability) and physical well-being for older adults; $NeuroPhone$, a mobile phone system to detect brain activity (P-300, blink) from off-the-shelf electroencephalography (EEG) headsets.\\
	2010 & {\bf Teaching Assistant} in Department of Computer Science at {\it Dartmouth College} \vspace{0.15cm}\\
			
			%&\underline{{\it Responsibility}}: I TAed this introductory course for computer science students. I helped set up assignments and gave lectures on a few classes.\\
	2008-2009 & {\bf Quantitative Software Developer} in {\it Stochastic Logic Ltd., Dhaka, Bangladesh}\\
			&  \underline{{\it Advisor}}: Arif Dowla, Ph.D. in Mathematics, UCSD\\	
			%& \underline{{\it Responsibility}}: Analysis, learning, and visualization of large financial time series data to aid decision making. I worked on projects that were outsourced to traders at Wall-street.\\
\end{longtable}
\vspace{0.5em}

%\newpage


\section*{\textbf{Publications}}
\setlength{\extrarowheight}{10pt}
\begin{comment}
\subsection*{\textbf{Under review}}
\vspace{-1.5em}
\begin{longtable}{p{0.7in}|p{5.5in}}
  2015 & \bibentry{mash2015ubicomp} \\
  	  & \bibentry{mash2015wh} \\ 
	  &\\
\end{longtable}
\end{comment}
\subsection*{\textbf{Peer reviewed Journals, Conference papers, and Book chapters}}
\vspace{-1.5em}
\begin{longtable}{p{0.7in}|p{5.5in}}
  2018 & \bibentry{rabbi2018cbp} \\ 
  	& \bibentry{rabbi2018saraprotocol} \\  
  2017 & \bibentry{rabbi2017towards} \\  
	  & \bibentry{choe2017semi} \\  
  2016 & \bibentry{aung2014leveraging} \\
  2015 & \bibentry{mash2015ubicomp} \\
  	  & \bibentry{mash2015wh} \\  
  	 & \bibentry{info:doi/10.2196/mhealth.4160}\\ 
  2014 & \bibentry{phil2014ph} \\
  	& \bibentry{lane2014bewell} \\
  2012 & \bibentry{lu2012stresssense} \\
  	& \bibentry{lin2012bewell+} \\
  2011 & \bibentry{rabbi2011passive} \\
       & \bibentry{lane2011bewell} \\
       & \bibentry{berke2011objective} \\
  2010 & \bibentry{campbell2010neurophone} \\
  	  & \bibentry{alamupright}\\
	  & \bibentry{alam2010minimum}\\
  2008 & \bibentry{alam2008upward}\\
\end{longtable}
%\newpage
\subsection*{\textbf{Lightly peer reviewed Abstracts, Posters, and Workshop papers}}
\vspace{-1.8em}
\begin{longtable}{p{0.7in}|p{5.5in}}
  2017 & \bibentry{rabbi2017sara} \\
  2015 & \bibentry{rabbisaint} \\
  2014 & \bibentry{mash2104stresscoping} \\
  2013 & \bibentry{voida2013moodrhythm} \\
  	& \bibentry{voida2013chiworkshop} \\
	& \bibentry{mash2013isbnpa} \\
	
\end{longtable}
\setlength{\extrarowheight}{5pt}


\begin{comment}
\vspace{2em}
\section*{\textbf{Skills}}
\vspace{-0.5em}
\begin{tabular}{>{\everypar{\hangindent0.5in}}p{6in}}
	\underline{Platforms}: Amazon Mechanical Turk, Android, WinBUGS, Web.py, D3\\
	\underline{Programming}: Matlab, R, C/C++, Java, Python, OpenGL, Shell Scripting, JavaScript, Latex \\
	\underline{Evaluation and methodology}: Randomized experiment design, Qualitative inquiry (daily diary study, semi-structured interviewing), Mixed-method, N of 1 evaluation\\
\end{tabular}
\end{comment}

\vspace{1em}
\section*{\textbf{Awards and grants}}
\vspace{-0.5em}
\begin{tabular}{>{\everypar{\hangindent0.5in}}p{6in}}
	- Pilot grant from Methodology Center at Penn State University, 2019 (in progress)\\
	- Pilot grant from Michigan Institute for Clinical \& Health Research based on SARA, 2018\\
	- Co-investigator of pilot grant from Translational Research Institute for Pain in Later Life on MyBehavior, 2015\\
	- Part of the winning team of  \$100K Heritage Open mHealth Challenge, 2013\\
	- Most helpful summer intern (among 60 student-interns) at AT\&T Labs Research, 2012\\
	- Dean's list on the year 2004 at Bangladesh Uni. of Engr. \& Tech.\\
\end{tabular}

\vspace{1em}
\section*{\textbf{Talks}}
\vspace{-0.5em}
\begin{longtable}{>{\everypar{\hangindent0.5in}}p{6in}}
	- \underline{SARA: Substance Abuse Research Assistant} at Prevention Center of \textit{Penn State University}, 2018\\
	- \underline{SARA: Substance Abuse Research Assistant} at Methodology Center of \textit{Society of Clinical Trials, Portland, Oregon}, 2018\\
	- \underline{SARA: Substance Abuse Research Assistant} at Methodology Center of \textit{Penn State University}, 2017\\
	- \underline{SARA: Substance Abuse Research Assistant} at Statistics PhD student seminar at \textit{Harvard University}, 2017\\
	- \underline{MyBehavior} at Methodology Center of \textit{Penn State University}, 2017\\
	- \underline{MyBehavior} at HCI Seminar of \textit{University of Rochester}, 2013\\
	- \underline{MyBehavior} at Information Science Brown-bag Series at \textit{Cornell University}, 2013\\
	- \underline{Mobile Phone Sensing} at the Computer Science Seminar at \textit{Bangladesh Uni. of Engr. \& Tech.}, 2012\\
	- \underline{Passive Assessment of Mental and Physical Well-being} at \textit{Ubicomp}, 2011\\
	- \underline{Passive Assessment of Mental and Physical Well-being} at Information Science Breakfast Series in \textit{Cornell University}, 2011\\
\end{longtable}

\section*{\textbf{Selected Press Coverage}}
\vspace{-0.7em}
%\usepackage{longtable}
\begin{longtable}{>{\everypar{\hangindent0.5in}}p{6in}}
	- \href{https://www.technologyreview.com/s/539721/a-health-tracking-app-you-might-actually-stick-with/}{A Health-Tracking App You Might Actually Stick With}, MIT Technology Review, 2015\\
	- \href{https://www.mobihealthnews.com/45795/cornell-researchers-use-personalized-algorithm-in-weight-loss-app/}{Cornell researchers use personalized algorithm in weight loss app}, MobiHealth News, 2015\\
	- \href{http://mashable.com/2015/07/30/health-tracking-app/#vsDHQmjzJ8qx}{This app helps you burn calories without radically changing your routine.}, Mashable, 2015\\
	- \href{http://www.att.com/gen/press-room?pid=23974}{Virtual Companion}, AT\&T Innovation Showcase ``Connecting Your World'', 2013\\
	- \href{http://www.economist.com/news/technology-quarterly/21578518-sensor-technology-microphones-are-designed-capture-sound-they-turn-out}{Teaching old microphones new tricks}, The Economist, 2013\\
	- \href{http://www.newscientist.com/article/mg21528775.400-smartphone-that-feels-your-strain.html}{Smartphone that feels your strain}, New Scientist, 2012\\
	- \href{http://phys.org/news/2012-08-voice-stress-software.html}{Voice-Stress Software Is Put to the Test}, PhysOrg and ACM Tech, 2012\\
	- \href{http://www.news.cornell.edu/stories/2011/10/your-phone-counselor-smartphone-monitors-stress}{Monitoring Mental Health from Your Pocket}, Cornell Chronicle, 2011\\
	- \href{http://www.nytimes.com/2011/09/18/magazine/the-cyborg-in-us-all.html?_r=2\&pagewanted=all}{Neural Phone is featured in The Cyborg in us all},  the NYTimes Magazine, 2011\\
	- \href{http://www.fastcoexist.com/1678760/get-some-therapy-from-an-app-that-reads-your-feelings-through-your-voice}{An App That Reads Your Feelings Through Your Voice}, Fast Co's Co.Exist piece, 2011\\
	- \href{http://earthsky.org/human-world/tanzeem-choudhury-develops-cellphone-apps-to-track-our-health}{Cellphone Apps to Track Our Health}, EarthSky, 2011\\
	- \href{http://www.technologyreview.com/view/418258/mobile-phone-mind-control/}{Mobile Phone Mind Control}, Technology Review, 2010\\
\end{longtable}

\vspace{0.7em}
\section*{\textbf{Students advised}}
\vspace{-0.5em}
\begin{longtable}{>{\everypar{\hangindent0.2in}}p{6in}}
	- Brian Lin, Undergraduate, Information Science, Cornell University, \textit{Fall 2011, Spring 2012, Fall 2012}\\
	- Jan Cardenas, Undergraduate, Information Science, Cornell University, \textit{Fall 2012}\\
	- Chantelle Farmer, Masters, Information Science, Cornell University, \textit{Spring 2013}\\
	- Thiago Caetano, Undergraduate visiting student from Universidade Estadual de Campinas at Cornell University, \textit{Summer 2013}\\
	- Max Schachere, High School student from Hawken School, Ohio, \textit{Summer 2013, Summer 2014}\\
	- Shankar Athinarayanan, Undergraduate, Computer Science, Cornell University, \textit{Fall 2013, Spring 2014, Fall 2014, Spring 2015}\\
	- Lily Gao, UX designer, Undergraduate, Information Science, Cornell University, \textit{Fall 2014}\\
	- Shreya Sitaraman, Undergraduate, Information Science, Cornell University, \textit{Summer 2014}\\
	- Jiaming Zhang, Undergraduate, School of Human Ecology, Cornell University, \textit{Summer 2015, Fall 2015}\\
	- Xian Zhang, Masters, Information Science, Cornell University, \textit{Spring 2015, Summer 2015, Fall 2015, Spring 2016}\\
	- Rohit Biswas, Undergraduate, Information Science, Cornell University, \textit{Summer 2015, Fall 2015, Spring 2016}\\
	- Minghao Li , Masters, Information Science, Cornell University, \textit{Summer 2015, Fall 2015, Spring 2016}\\
	- Zhe Lin , Masters, Information Science, Cornell University, \textit{Summer 2015, Fall 2015, Spring 2016}\\
	- Kelly Hall, Undergraduate, Statistics, Unviersity of Michigan, \textit{Fall 2016}\\
	- Bess Rothman, Undergraduate, Statistics, Unviersity of Michigan, \textit{Summer 2017, Fall 2017}\\
	- Katherine Li, Undergraduate, Statistics, Unviersity of Michigan, \textit{Fall 2016, Spring 2017, Summer 2017, Fall 2017, Spring 2018}
\end{longtable}

\vspace{1em}
\section*{\textbf{Reference}}
\vspace{-0.5em}
\begin{longtable}{>{\everypar{\hangindent0.5in}}p{6in}}
	- Tanzeem Choudhury, Associate Professor at Department of  Information Science, Cornell University\\
	- Susan Murphy, Professor at Department of Statistics and Computer Science, Harvard University\\
	- Andrew Campbell, Professor at Computer Science, Dartmouth College\\
	- Maureen Walton, Professor at Department of Psychiatry, University of Michigan, Ann Arbor\\
	- Predrag Klasnja, Assistant Professor at School of Information Science, University of Michigan, Ann Arbor
	%- Mi Zhang, Assistant Professor at Dept. of Electrical and Computer Engineering, Michigan State University
\end{longtable}



\begin{comment}
\vspace{0.7em}
\section*{\textbf{Selected Current and Past Collaborators}}
\vspace{-0.7em}
\begin{tabular}{>{\everypar{\hangindent0.5in}}p{6in}}
	Andrew Campbell, Professor at Dartmouth College, 2010-now\\
	Deborah Estrin, Professor at Cornell Tech, 2012-now\\
	Ethan Berke, Associate Professor at Dartmouth Medical School, 2009-2011\\
	Erica Phillips Caesar, Associate Professor at Cornell Weil Medical School, 2013-now\\
	Emiliano Miluzzo, Senior Member of Technical Staff at AT\&T Labs Research, 2012-now\\
	Nicholas Lane, Lead Researcher in Microsoft Research Asia, 2010-2012\\
	Mi Zhang, Postdoc at Cornell University, 2013-now\\
	Mark Mattews, Postdoc at Cornell University, 2013-now\\
	Erin Caroll, Postdoc at University of Rochester, 2013-now\\
	Hong Lu, Researcher Scientist at Intel Labs, 2010-2012\\	
	Amy Cuddy, Associate Professor at Harvard Business School, 2012\\
	Daniel Gatica-Perez, EPFL, 2011-2012
\end{tabular}
\end{comment}


\begin{comment}
\section*{\textbf{Service}}
\vspace{-0.5em}
\begin{tabular}{>{\everypar{\hangindent0.5in}}p{6in}}
	\underline{Reviewer}: Ubicomp 2015, CHI 2015, CHI 2014, Ubicomp 2014, UIST 2014, CHI 2013, Mobicase 2013, Ubicomp 2013, Ubicomp 2012, Pervasive 2012, ACII 2011, International Journal of Distributed Sensor Networks, Pervasive and Mobile Computing special issue on Nokia Mobile Data Challenge, ACM Transactions on Interactive Intelligent Systems\\
	\underline{Program Committee}: Affective Computing and Intelligent Interaction (ACII 2011), \href{http://what2013workshop.wordpress.com/2013/05/29/what-2013/}{Workshop on Human And Technology (WHAT 2013)}\\
\end{tabular}


\vspace{2em}
\section*{\textbf{Selected Graduate Courses}}
\vspace{-0.5em}
\begin{tabular}{>{\everypar{\hangindent0.5in}}p{6in}}
	\underline{Computer Science}: Artificial Intelligence, Machine Learning, Probabilistic Graphical Models, Numerical Linear Algebra, Natural Language Processing, Computational Social Science, Networks, Decision Theory\\
	\underline{Others}: Qualitative Methods; Quantitative Methods (experiment design) in Psychology; Human-Computer Interaction; Theories of Information, Technology and Society\\
\end{tabular}
\end{comment}




% Footer
\bigskip
\begin{center}
  \begin{footnotesize}
    Last updated: \today
  \end{footnotesize}
\end{center}

\nobibliography{cv.bib}
\end{document}
